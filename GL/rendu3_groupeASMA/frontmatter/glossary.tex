\makeglossaries
\thispagestyle{glossary}

\newglossaryentry{hypermarche} {
    name=hypermarché,
    description={un hypermarché est un commerce de détail en libre-service, de grande taille (définie en France par une surface égale ou supérieure à 2 500 m2.}
}

\newglossaryentry{douchette} {
	name=douchette,
	description={représente un lecteur de code-barres, permet de lire les informations stockées sous la forme de codes-barres ou QRCode.}
}

\newglossaryentry{code_barres} {
	name=code-barres,
	description={représente une donnée numérique ou alphanumérique sous forme d'un symbole constitué de barres et d'espaces dont l'épaisseur varie en fonction de la symbologie utilisée et des données ainsi codées.}
}

\newglossaryentry{qr_code} {
	name=QR Code,
	description={un type de code-barres en deux dimensions constitué de modules noirs disposés dans un carré à fond blanc. L'agencement de ces points définit l'information que contient le code. QR (abréviation de Quick Response) signifie que le contenu du code peut être décodé rapidement après avoir été lu par un lecteur de code-barres, un téléphone mobile, un smartphone, ou encore une webcam. Son avantage est de pouvoir stocker plus d'informations qu'un code à barres, et surtout des données directement reconnues par des applications, permettant ainsi de déclencher facilement des actions.}
}

\newglossaryentry{gamme} {
	name=gamme,
	description={une gamme de produits est généralement définie comme un ensemble de produits de même catégorie ou répondant au même type de besoin proposé par une même marque ou fabricant.}
}

\newglossaryentry{smartphone} {
	name=smartphone,
	description={un smartphone, ordiphone ou téléphone intelligent, est un téléphone mobile évolué disposant des fonctions d'un assistant numérique personnel, d'un appareil photo numérique et d'un ordinateur portable. Il peut exécuter divers logiciels/applications grâce à un système d'exploitation}
}

\newglossaryentry{caddie} {
	name=caddie,
	description={un chariot, métallique ou en matière plastique, conçu pour faciliter le transport des marchandises achetées dans un supermarché ou d'autres types de magasins. (remarque : Caddie est une marque déposé par une société anonyme patronyme)}
}

\newglossaryentry{robot} {
	name=robot,
	description={Un robot est un dispositif mécatronique (alliant mécanique, électronique et informatique) accomplissant automatiquement soit des tâches qui sont généralement dangereuses, pénibles, répétitives ou impossibles pour les humains, soit des tâches plus simples mais en les réalisant mieux que ce que ferait un être humain.}
}

\newglossaryentry{tablette_interactive} {
	name=tablette interactive,
	description={fait référence à une tablette de grande taille (environ la surface d’un téléviseur de salon). Un ordinateur portable ultraplat qui se présente sous la forme d'un écran tactile sans clavier et qui offre à peu près les mêmes fonctionnalités qu'un ordinateur personnel.}
}

\newglossaryentry{Wifi} {
	name=Wifi,
	description={Le Wifi est un ensemble de protocoles de communication sans fil régis par les normes du groupe IEEE 802.11 (ISO/CEI 8802-11). Un réseau Wifi permet de relier sans fil plusieurs appareils informatiques (ordinateur, routeur , smartphone, décodeur Internet, etc.) au sein d'un réseau informatique afin de permettre la transmission de données entre eux.}
}

\newglossaryentry{GPS} {
	name=GPS,
	description={Le Global Positioning System (GPS) – que l'on peut traduire en français par « système de localisation mondial » – est un système de géolocalisation fonctionnant au niveau mondial. En 2011, il est avec GLONASS, un système de positionnement par satellites entièrement opérationnel et accessible au grand public.}
}

\newglossaryentry{NFC} {
	name=NFC,
	description={La communication en champ proche (en anglais near field communication, NFC) est une technologie de communication sans-fil à courte portée et haute fréquence, permettant l'échange d'informations entre des périphériques jusqu'à une distance d'environ 10 cm. Cette technologie est une extension de la norme ISO/CEI 14443 standardisant les cartes de proximité utilisant la radio-identification (RFID), qui combinent l'interface d'une carte à puce et un lecteur au sein d'un seul périphérique. Un périphérique NFC est capable de communiquer avec le matériel ISO/CEI 14443 existant, avec un autre périphérique NFC ou avec certaines infrastructures sans-contact existantes comme les valideurs des transports en commun ou les terminaux de paiement chez les commerçants.}
}
