\subsection{Scénarios concrets}
\subsubsection{Avec un \gls{smartphone} à l'\gls{hypermarche}}
					José va faire ses courses dans l'\gls{hypermarche} du futur. Il s'arme de son \gls{smartphone} et lance l'application du futur, nommée fDrive.\\
					À l'entrée, il se connecte au réseau \gls{Wifi} de l'\gls{hypermarche} et obtient toutes les nouvelles promotions et nouveaux arrangements du magasin. Puisqu'il a l'habitude d'acheter les mêmes articles chaque semaine, José consulte son historique d'achats et sélectionne ce qu'il désire se procurer à nouveau.\\
					Il veut aussi tester une nouvelle \gls{gamme} de shampooing "spéciale crépus" qui vient de faire irruption dans le magasin; il tape le mot "shampooing" dans la barre de recherche de son application et se laisse guider vers le rayon Produits de beauté. L'application fait en sorte de le faire passer par quelques promotions sur des articles déjà achetés auparavant.\\
					Ignorant ces promotions, José court vers le shampooing qu'il désire et flashe son article. La description et le prix de celui-ci s'affichent. Il ajoute le shampooing à son panier et valide ce dernier. L'application lui indique l'endroit où il doit payer. Il se fait guider, à la manière d'un \gls{GPS}, au guichet indiqué.\\
					2 minutes d'attente plus tard, son numéro de commande apparaît sur le guichet. Il paye avec son téléphone et récupère son \gls{caddie} préparé avec soin par un \gls{robot}. Heureux, il repart de l'hypermarché du futur avec la plus grande des satisfactions.
				\subsubsection{Sans \gls{smartphone} à l'\gls{hypermarche}}
					Dartagnan veut faire ses courses. Il a 63 ans, en pleine forme et n'a pas de \gls{smartphone}. Il ne sait pas ce qui il veut acheter, il veut juste voir à quoi ressemble l'\gls{hypermarche} du futur dont tout le monde parle tant.\\
					Il entre dans le magasin et une hôtesse l'accueille pour lui en expliquer le fonctionnement. Comme il n'a pas de \gls{smartphone}, elle lui propose d'utiliser une \gls{douchette} du magasin, nommée fDouchette, pour flasher les articles avec.
					Elle l'accompagne dans le magasin pour lui montrer comment se servir de cette \gls{douchette} : il flashe un article, il peut choisir la quantité, et il peut aussi le supprimer s'il se trompe. Elle lui explique qu'il n'a pas besoin de prendre les articles et les mettre dans un \gls{caddie} car un \gls{robot} se charge de prendre tous les articles pour lui et les lui ramener à la caisse. Dartagnan est content parce qu'il n'a rien à transporter.\\
					Il demande ensuite à l'hôtesse où il peut se procurer des cookies d'une certaine marque, un article qui est devenu rare. L'hôtesse l'ignore mais en profite pour l'emmener voir une des tables interactives disposées dans le magasin tous les 25m. La table permet de rechercher un article, la prix, la description, l'endroit où il peut le trouver dans le magasin, et enfin s'il y a des promos sur les produit similaires. Il peut directement le scanner sur la table grâce  au QR code.\\
					Mais comme il aime voir les articles de ses propres yeux, l'hôtesse hui montre qu'il peut télécharger un plan sur sa \gls{douchette} pour aller voir l'article. Il télécharge le plan, le suit et arrive rapidement à l'article, puis il le scanne et l'article est ajouté dans son panier.\\
					Il décide enfin de passer à la caisse. Alice, l'hôtesse, lui indique comment valider ses articles en appuyant sur le bouton "caisse" de sa \gls{douchette}, et lui montre un plan pour aller à la caisse.
Il va à la caisse et attend qu'on appelle son numéro de commande; il peut payer par espèce ou carte bleue. Alice lui propose alors de se faire livrer ses articles chez lui. Il lui suffit d'appuyer sur livraison à domicile sur la \gls{douchette} et entre l'adresse et l'heure de livraison.\\
Il le fait, rentre chez lui et attend l'arrivée de ses articles. Dartagnan est aux anges : faire des courses n'a jamais été aussi plaisant et reposant pour lui.
				\subsubsection{Commande en ligne et livraison à domicile}
					Bob, ingénieur informatique dans une grande société internationale, prend sa pause de 16H30. Durant cette pause, il décide de commander ses courses hebdomadaires via le site de l'\gls{hypermarche} fDrive sur son ordinateur au bureau.\\
					Il se connecte sur son compte. Bob, n'étant pas quelqu'un de compliqué, il décide de commander ce qu'il a commandé la semaine dernière via la fonctionnalité "Historique" de l'application. Il veut aussi ajouter des pâtes à sa commande. Il recherche le mot "pâtes" dans la barre de recherche et l'application affiche une liste de pâtes avec leur prix. Il en sélectionne un et l'ajoute à sa commande. Il remarque aussi une publicité d'un nouveau jeu vidéo et l'ajoute à son panier.\\
Il consulte son panier pour voir si tout ce qu'il veut acheter est là. Il change la quantité de riz car il n'en avait pas pris assez la dernière fois. Il valide ensuite sa commande. Il n'a pas envie de passer à l'\gls{hypermarche} pour prendre ses courses et décide de se faire livrer ses articles. Il sélectionne son adresse personnelle qui est déjà associée à son compte. Il choisit comme date et heure de livraison le soir même à 19H. Comme il habite dans un rayon de 20km du centre de distribution de marchandise et qu'il n'y a pas d'article lourd dans sa commande, la livraison est gratuite. \\
Il paie en ligne en entrant son numéro de carte bleue et les informations de sécurité. Il reçoit la confirmation de paiement. Il peut suivre l'avancement de sa commande en ligne.
Il arrive chez lui à 18H55 et 5 minutes plus tard, un livreur vient apporter sa commande.
				\subsubsection{Commande en ligne sur \gls{smartphone} et Drive}
					Bob, ingénieur informatique dans une grande société internationale, prend sa pause de 16H30. Durant cette pause, il décide de faire ses courses hebdomadaires via l'application fDrive. Finissant à 18H, il indique vouloir prendre sa commande vers 18H30, sur la route du retour.\\
Il est maintenant 18H et Bob se met en route, il arrive à l'\gls{hypermarche} du futur avec une avance de 5 minutes. Il lance son application fDrive et indique sa présence sur les lieux de chargements. Super ! Sa commande est déjà prête et l'application lui indique la voie vers laquelle il doit se diriger. Pour gagner du temps, il décide de payer avec son \gls{smartphone}.\\
Une fois sur la voie et quelques minutes d'attente plus tard, c'est son tour. Sans même sortir de sa voiture, un \gls{robot} vient lui charger ses articles. Voilà, les courses sont faites, il n'y a plus qu'à rentrer chez soi.
				\subsubsection{Commande d'un produit inconnu avec \gls{qr_code}}
					Mélanie, étudiante en psychologie, passe sa soirée à réviser chez une amie. Après deux heures intensives de remue-méninges, son amie lui propose un petit rafraîchissement : un soda d'une marque inconnue que Mélanie n'avait jamais vu ou goûté auparavant. Intriguée, elle décide de se laisser tenter. Surprenant ! Elle adore !\\
					Comme par instinct, elle décide de sortir son \gls{smartphone} et de lancer l'application fDrive, elle utilise la fonctionnalité "Scan", et scanne le QRCode de la canette mystère. Par chance, il se trouve que l'\gls{hypermarche} du futur possède ce rafraîchissement.\\
					Elle en commande directement un pack de douze, paie en ligne, et se fait livrer à domicile le lendemain. Désormais, les révisions ne seront plus qu'une partie de plaisir.
				\subsubsection{Commande d'un produit inconnu avec une photo}
					Mélanie, étudiante en psychologie, passe sa soirée à réviser chez son amie Anne. Après deux heures de révisions intensives, elle remarque le joli pull de son amie. Un pull d'une marque inconnue que Mélanie n'avait jamais vu auparavant. Intriguée, elle décide de lui demander où elle l'a acheté. Son amie ne sait pas car elle l'a eu en cadeau.\\
					Comme par instinct, Mélanie décide de sortir son \gls{smartphone} et de lancer l'application fDrive. Elle prend en photo le pull et utilise la fonctionnalité "Reconnaissance de produit". Après quelques secondes de recherche,  l'application confirme que l'\gls{hypermarche} du futur possède ce pull et affiche les informations sur ce produit.\\
					Elle en commande directement deux, paie en ligne, et se fait livrer à domicile le lendemain.
				\subsubsection{Appel téléphonique et promotions}
					David, ingénieur dans une entreprise, 40 ans. Sa femme, Edith, l'appelle à midi pour lui dire qu'il manque certaines choses chez eux. Il faudrait qu'il fasse les courses mais il est occupé tout l'après-midi et il sort du travail tard.\\
					Pour ne perdre pas du temps il décide d'appeler l'\gls{hypermarche} du futur. Un répondeur lui dit qu'il a la chance parce que aujourd'hui il y a des promotions s'il commande plus de 4 bouteilles des lait. Il commande tous les articles que sa femme avait demandés. Il décide de rentrer chez lui après le boulot et d'attendre la livraison, mais le répondeur le rappel quelque instants plus tard pour lui indiquer qu'il peut payé 10\% moins cher s'il passe après 20H pour récupérer sa commande. Il décide d'opter pour cette solution.\\
					Possédant un compte client dans le supermarché, il décide de payer par téléphone. Seuls son identifiant et mot de passe sont nécessaires. Enfin, il décide de passer à 20H. Une fois sur le parking, il n'a pas besoin de se garer, un \gls{robot} l'attend, ouvre le coffre et charge tous ses articles. Il rentre chez lui heureux.
				\subsubsection{Articles non disponibles}
					Ben, 18 ans veut acheter faire ses courses de la semaine le samedi matin. Il a beaucoup de temps et veut faire de l'exercice en marchant dans le supermarché.\\
					Il utilise son \gls{smartphone} pour flasher les articles. Il parcourt tous les rayons sans utiliser la carte du supermarché. Il ne retrouve pas deux articles après avoir parcouru tous les rayons. Il utilise alors l'application fDrive pour les rechercher. Les produits qu'il recherche ne sont pas actuellement en stock au supermarché. L'application lui propose alors de commander les articles et de se les faire livrer gratuitement le soir. Comme son adresse est enregistrée sur fDrive, il n’a rien à entrer. Il décide de les commander et passe ensuite à la caisse. Il paie tous les articles qu'il a achetés et commandés. Il rentre chez lui.\\
L'après-midi, le supermarché renouvelle son stock d'articles. Les articles que Ben a commandés lui sont envoyés. Le soir, un livreur de l'\gls{hypermarche} du futur sonne chez Ben et lui donne les articles.
				\subsubsection{Commande d'anniversaire}
					Hubert, 27 ans, jeune doctorant. Hubert sort tout juste du laboratoire un soir de la semaine, et se rappelle que l'anniversaire de son petit frère, Pierre, tombe la semaine prochaine. Hélas il ne pourra pas être présent pour cause d'une conférence à Dali.\\
					Il décide donc de passer à l'\gls{hypermarche} du futur. Une fois sur place, il trouve un cadeau adéquat pour son petit frère. Il passe également devant une tarte au citron meringué et se rappelle que Pierre adore ce genre de gâteau. Une fois à la caisse et les deux articles flashés à l'aide de son \gls{smartphone}, il décide de les faire livrer le jour de l'anniversaire de Pierre, le tout accompagné d'un message de souhaits d'anniversaire.\\
					Il paye en avance. Il peut maintenant partir à Dali serein, son petit frère aura bien un cadeau et un gâteau d'anniversaire, même s'il ne sera pas présent.
				\subsubsection{Achat de produits frais}
					Jihane est une jeune mariée et diplômée en informatique. Son époux, Kader, est victime d’une fièvre un Samedi de Juillet et a besoin de fruits frais pour reprendre des forces. Jihane n’hésite pas une seconde et se rend à l’hypermarché du futur.  Étant habituée à faire ses courses dans ce lieu novateur, elle n’a pas besoin d’aide humaine ou matérielle pour retrouver les fruits qu’elle cherche (hôtesse, smartphone, tablette interactive…) car elle connaît le magasin comme sa poche. Elle demande quand meme l’assistance d’un robot grace à son smartphone pour l’aider à transporter ses fruits. L’application la localise et après quelque seconde lui envoie un robot pour l’aider. Elle se dirige ensuite comme par instinct vers le rayon des produits frais, qui sont exposés en rayon et acoompagné de son robot. Elle prend son temps pour sélectionner les mangues, pêches et bananes les plus mûres. De nature très méticuleuse, elle prend 10 minutes pour trouver son plaisir, ne désirant que le meilleur pour son homme. Satisfaite, elle met ses articles dans son caddie qui est transporté par le robot et se dirige vers les caisses classiques. Le robot lui pèse ses fruits et les scanne. Contente de ses courses et surexcitée à l’idée de guérir son mari, Jihane sort de l’hypermarché en sifflotant et file vers sa voiture, le robot l’accompagnant avec ses courses. Le robot l’aide aussi à mettre ses course dans la voiture puis il revient vers l’hypermarché et Jihane s’en va.
